\subsubsection{Lizenzen}
\label{subsubsec:lizenzen}

Die verwendeten Librarys sind grundsätzlich aus der Arduino IDE von Arduino. Arduino selbst ist eine aus Soft- und Hardware bestehende Physical-Computing-Plattform, bei welcher beide Komponenten auf Open-Source-Basis quelloffen sind \cite{arduinoWiki}. Nach Aussage von Arduino selbst, stehen alle C/C++ Mikrocontroller Librarys unter der LGPL \cite{ArduinoLicense2019}. Alle Lizenzen, sowie auch die Autoren wurden aus dem Text in den Librarys selbst entnommen. All diese Lizenz- und Copyright-Texte wurden in den dementsprechenden Klassen oben hinzugefügt.\\

\begin{table}[h]
\centering
\caption{Lizenzen}
\label{tab:lizenzen}
\begin{tabular}{|l|l|l|}
\hline 
\textbf{Library} & \textbf{Autor} & \textbf{Lizenz} \\ 
\hline 
Arduino & Arduino Team & LGPL V2.1 \\ 
\hline 
SPI & Cristian Maglie, Paul Stoffregen, Matthijs Kooijman, Andrew J. Kroll & LGPL V2.1 \\ 
\hline 
Wire & Nicholas Zambetti, Todd Krein, Chuck Todd & LGPL V2.1 \\ 
\hline 
SoftwareSerial & Limor Fried, Mikal Hart, Paul Stoffregen, Garrett Mace, Brett Hagman & LGPL V2.1 \\ 
\hline 
SD & SparkFun Electronics & GPL V3 \\ 
\hline 
RTClib & JeeLabs & - \\ 
\hline 
Adafruit\_Sensor & Kevin Townsend & Apache V2.0 \\ 
\hline 
Adafruit\_BME280 & Kevin Townsend & BSD \\ 
\hline 
Adafruit\_FONA & Limor Fried & BSD \\ 
\hline 
Adafruit\_TSL2561\_U & Kevin Townsend & BSD \\ 
\hline 
\end{tabular} 
\end{table}

Für die RTClib von JeeLabs wurde keine Lizenz gefunden. Es steht im Text lediglich \glqq \textit{Released to the public domain! Enjoy!}\grqq\,(siehe Klasse \_DS3231\_). Die SD-Library hat eine GPL weil auch die darin enthaltene \textit{sdfatlib} unter der GPL steht.\\
