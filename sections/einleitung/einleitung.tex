\subsection{Einleitung}
\label{subsec:einleitung}
Pflanzen benötigen eine ihnen entsprechende Umwelt. Durch meteorologische Messdaten kann diese ermittelt und durch Agronome optimal bewirtschaftet werden. Ausserdem können bei genügend Messdaten aus einem Erfassungsnetz Wetterprognosen erstellt werden. Die Erhebung solcher Messdaten trägt somit erheblich zum wirtschaftlichen Erfolg in der Agronomie bei und kann bei einem geeignet grossen Erfassungsnetz Bewohner vor Unwetter warnen. In den ärmeren Teilen Afrikas und Südamerikas sind solche Systeme jedoch nicht verbreitet. Aus diesem Grund wird eine kostengünstige, erweiterbare und mobile Wetterstation realisiert, welche die örtlichen Agronomen unterstützen kann. Diese Wetterstation kann die Niederschlagsmenge, die Windstärke \& -richtung, die Lufttemperatur, -druck, -feuchtigkeit und die Sonnenstunden messen. Außerdem kann der Akkumulator der Wetterstation mittels Photovoltaik geladen werden. Die erhobenen Daten sind via SMS\footnote{Mittels GSM/GPRS-Modul} abrufbar und mit einem GPS-Modul kann der Standort der mobilen Wetterstation erfasst werden. Zusätzlich kann mittels eines Terminals (z.B. PuTTY) eine Verbindung über eine serielle Schnittstelle über den USB 2.0 micro-B Anschluss der Wetterstation erstellt und mit ihr kommuniziert werden.\\

Mittels Ombrometer mit Kipplöffelprinzip wird die Niederschlagsmenge ermittelt. Die Windstärke wird über ein Schalenkreuzanemometer und die Windrichtung über einen Windrichtungsgeber eruiert. Über einen RJ-11 Stecker werden diese dann an die Wetterstation angeschlossen und die analogen Signale von einer Mikrocontrollereinheit ausgewertet. Die Lufttemperatur, -druck und -feuchtigkeit werden über den digitalen Kombinationssensor BME280 von \textit{Bosch} und die Beleuchtungsstärke (Lux) zur Ermittlung der Sonnenstunden über den Light-To-Digital Converter TSL2561 von \textit{TAOS}\footnote{Texas Advanced Optoelectronic Solutions} gemessen. Um den Messdaten einen Zeitstempel beifügen zu können, wird die Extremely Accurate Real Time Clock DS3231 von \textit{Maxim Integrated} verwendet. Diese drei digitalen Elemente sind über das I$^{2}$C-Interface mit der Mikrocontrollereinheit verbunden. Zur Datenspeicherung wird eine $\mu$SDHC-Karte verwendet. Im Kommunikationsmodul SIM808 von \textit{SIM Com} werden das GSM/GPRS- und GPS-Modul vereint.\\

Die (Teil-)Realisierung\footnote{Es wird in weiterführenden Projekten weiter daran gearbeitet} dieser Wetterstation wurde in zwei Projekten an der Hochschule für Technik der FHNW im Studiengang Elektro- und Informationstechnik durchgeführt. Diese bestehen aus einem vorangehenden Projekt 5, sowie dem anschließenden Projekt 6 (Bachelor Diplomarbeit) und wurden vom gleichen Team bearbeitet. Dieser Bericht umfasst die Inhalte beider Projekte. Welche Aspekte in welchem Projekt behandelt wurden, können dem Kapitel \ref{subsec:Ziele} \nameref{subsec:Ziele} direkt entnommen werden.\\

\todo[inline]{Berichtaufbau}
