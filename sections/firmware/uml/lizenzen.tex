\subsubsection{Lizenzen}
\label{subsubsec:lizenzen}

Die verwendeten Librarys sind grundsätzlich aus der Arduino IDE von Arduino. Arduino selbst ist eine aus Soft- und Hardware bestehende Physical-Computing-Plattform, bei welcher beide Komponenten auf Open-Source-Basis quelloffen sind \cite{arduinoWiki}. Nach Aussage von Arduino selbst, stehen alle C/C++ Mikrocontroller Librarys unter der \textbf{LGPL} \cite{ArduinoLicense2019}. Einige Lizenz-Texte stehen im Anhang \ref{sec:lizenztexte}. \\

%Die RTClib steht unter der \textbf{MIT}-Lizenz, zusätzlich befindet sich noch der Lizenz-Text im Anhang \ref{subsec:rtclib_lizenztext}. Auch der Lizenz-Text der Adafruit_BME280 ist im Anhang \ref{subsec:adafruit_bme280_lizenztext}.\\

\begin{table}[h]
\centering
\caption{Lizenzen}
\label{tab:lizenzen}
\begin{tabular}{|l|l|l|}
\hline 
\textbf{Library} & \textbf{Author} & \textbf{Lizenz} \\ 
\hline 
Arduino & • & • \\ 
\hline 
SPI & • & • \\ 
\hline 
Wire & • & • \\ 
\hline 
SoftwareSerial & • & • \\ 
\hline 
SD & • & • \\ 
\hline 
RTClib & • & • \\ 
\hline 
Adafruit\_Sensor & • & • \\ 
\hline 
Adafruit\_BME280 & • & • \\ 
\hline 
Adafruit\_FONA & • & • \\ 
\hline 
Adafruit\_TSL2561\_U & • & • \\ 
\hline 
\end{tabular} 
\end{table}