\subsection{Validierung der Firmware}
\label{subsec:validierung_Firmware}
Die Validierung der Firmware verläuft in verschiedenen Abschnitten. Zuerst werden die Subsysteme, rsp. die einzelnen Klassen (siehe UML-Diagramm Abb. \ref{fig:uml_diagramm}) auf ihre Funktionstüchtigkeit getestet. Diese Validierung wurde meist bereits bei der Implementierung direkt erledigt, wobei grundsätzlich die Daten über die serielle Schnittstelle herausgeschrieben wurden (siehe Kapitel \ref{subsubsec:usbtocomputer} \nameref{subsubsec:usbtocomputer}). Zum Schluss dann nochmals die Firmware als ganzes. Dabei wird überprüft, ob die Klassen richtig miteinander funktionieren.\\

Im Allgemeinen lässt sich sagen, dass es bei aufgetauchten Problemen teils nicht direkt eruieren lässt, ob es ein Firm- oder Hardware Problem ist. Also ist das Troubleshooting recht aufwendig.\\


\subsubsection{\_BME280\_}
\label{subsubsec:val_BME280}

\subsubsection{\_DS3231\_}
\label{subsubsec:valDS3231}

\subsubsection{\_TSL2560\_}
\label{subsubsec:valTSL2560}

\subsubsection{\_SIM808\_}
\label{subsubsec:valSIM808}

\subsubsection{Anemometer}
\label{subsubsec:valAnemometer}

\subsubsection{Ombrometer}
\label{subsubsec:valOmbrometer}

\subsubsection{CommandLineInterface}
\label{subsubsec:valCommandLineInterface}

\subsubsection{SDCard}
\label{subsubsec:valSDCard}

\subsubsection{Windrichtungsgeber}
\label{subsubsec:valWindrichtungsgeber}

\subsubsection{Gesamt}
\label{subsubsec:valGesamt}